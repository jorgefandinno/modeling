% ---------------------------------------------------------------------
\begin{frame}{Simplified STRIPS\footnote{Stanford Research Institute Problem Solver, 1971} Planning}

  \begin{itemize}
  \item<2-> \structure{Problem Instance}
    \begin{itemize}
    \item set of fluents
    \item initial and goal state
    \item set of actions, consisting of pre- and postconditions
    \item number $k$ of allowed actions
    \end{itemize}
  \item<2-> \structure{Problem Class}
    Find a plan, that is, a sequence of $k$ actions leading from the initial state to the goal state
    \smallskip
  \item<3-> \structure{Example}
    \begin{itemize}
    \item fluents       \ $\{p,q,r\}$
    \item initial state \ $\{p\}$
    \item goal state    \ $\{r\}$
    \item actions       \ $a = (\{p\},\{q,\neg p\})$ and $b = (\{q\},\{r,\neg q\})$
    \item length        \ 2
      \medskip
    \item<4-> plan      \ $\langle a, b \rangle$
    \end{itemize}
    \medskip
  \end{itemize}

\end{frame}
% ---------------------------------------------------------------------
\begin{frame}[fragile,shrink=1]{Simplistic STRIPS Planning}
\begin{semiverbatim}
time(1..k).
\pause
fluent(p).     action(a).     action(b).       init(p).
fluent(q).        pre(a,p).      pre(b,q).
fluent(r).        add(a,q).      add(b,r).     query(r).
                  del(a,p).      del(b,q).
\pause
holds(P,0) :- init(P).

\{ occ(A,T) : action(A) \} = 1 :- time(T).
:- occ(A,T), pre(A,F), not holds(F,T-1).

holds(F,T) :- occ(A,T), add(A,F).
holds(F,T) :- holds(F,T-1), time(T), not occ(A,T) : del(A,F).
\pause
:- query(F), not holds(F,k).
\end{semiverbatim}
\end{frame}
% ------------------------------------------------------------
%
%%% Local Variables:
%%% mode: latex
%%% TeX-master: "../../main"
%%% End:
